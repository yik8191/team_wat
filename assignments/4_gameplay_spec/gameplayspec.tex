\documentclass[]{article}
\usepackage{amssymb,latexsym,amsmath}     % Standard packages
\usepackage[pdftex]{graphicx}
\usepackage{indentfirst}
\usepackage{pbox}
\usepackage{array}
\renewcommand{\baselinestretch}{1.05} 
\newcolumntype{L}[1]{>{\raggedright\let\newline\\\arraybackslash\hspace{0pt}}m{#1}}


\addtolength{\textwidth}{1.0in}
\addtolength{\textheight}{1.00in}
\addtolength{\evensidemargin}{-0.75in}
\addtolength{\oddsidemargin}{-0.75in}
\addtolength{\topmargin}{-.50in}
\usepackage[lf]{venturis}
% \usepackage[T1]{fontenc}
% \usepackage{newtxtext,newtxmath}
% \renewcommand{\familydefault}{\sfdefault}  % sans-serif!

\usepackage{fancyhdr}
\renewcommand\headrule{}
\pagestyle{fancy}
\makeatletter
\let\ps@plain\ps@fancy % make plain pages fancy
\makeatother
\fancyhf{}
\lhead{\textbf{Rhythm Knights}\\
Gameplay Specification}
\rhead{\resizebox{0.75in}{!}{\includegraphics{img/logo_team.jpg}}}
\rfoot{\thepage}

\begin{document}

\title{\textbf{Rhythm Knights} \\ Gameplay Specification}
\author{Team \emph{wat} \\
\small{Kylar Henderson, Charles Tark, 
Gagik Hakobyan, Julia Cole, Andrew Halpern, Austin Liu}}
\date{} % removed date
\maketitle

\section*{Core Vision}
The main character arrives at a costume party fashionably late and
realizes something is amuck. All of the other party attendees have
been turned into the monsters they were dressed as by the evil DJ! The
main character must traverse the dance floor through hordes of
monsters to reach the DJ and save them all with the power of
music. Players must solve challenging environmental puzzles as they
perform rhythm based actions to the beat of the dance music. Along the
way they will discover useful new abilities and tools that add an
additional layer of complexity to each floor.

\section*{Design Philosophy}

Our game is focused on creating a wacky costume dance party with
possessed enemy characters taking on the role of their costumes. The
dastardly DJ is trying to control the crowd, while the main
character's goal is to match the fast-paced techno-music of the DJ to
conquer the dance floor. Players must make every action on the beat of
the music and think fast in order to calculate one of the paths that
lead to the dance floor exit.\\

The combination of combat and rhythm elements makes this puzzle game
timing-oriented. Being forced to restart a level is part of the
learning experience in the game. The 'rhythm ticker' is an essential
guide for the player, but players can experiment with a different
combination of attacks and maneuvers to conquer the dance floor.\\

As a puzzle game, our designs will be clean and appealing. A colorful,
whimsical setting with quirky animations will keep the player engaged
not only mentally, but visually as well. The environment of the game
will be responsive to the beat of the music and will reflect the
musical tone and pattern.

\section*{Objectives}

The main objective for each dance floor is to match the actions
displayed on the rhythm action bar. If players successfully fulfill
all the actions required on the action bar players will be able to
move onto the next level. If players cannot match all the actions in
the order given players will be forced to reset the level. Also,
players must perform these actions on the beat of the music otherwise
the player will be forced to restart. If there is no action specified
on the rhythm action bar players are free to perform any action or
maneuver they wish; however, if there are no enemies on the map left
and there are still attack actions specified players must restart the
level before exiting the dance floor.


\section*{Actions}
\begin{table}[h]
\begin{tabular}{|l|L{2cm}|L{4cm}|L{4cm}|l|}
\hline
\textbf{Verb} & \textbf{Input} & \textbf{Outcome}  & \textbf{Limitations} & \textbf{Importance} \\
\hline
Move & WASD  & Player attempts to moves one space in specified direction 
 & Must be done on rhythm, cannot move onto obstacle tile & Critical   \\
\hline
Jump & Double Tap WASD & Player attempts to moves two spaces in specified direction 
 & Must be done on rhythm, cannot jump through obstacles & Critical   \\
\hline
Cast Freeze & F & Freeze all enemies for three beats   & Must be done on rhythm, can only be used once per level & Critical   \\
\hline
Shoot & Space & Shoot in a straight line & Must be done on rhythm, can only shoot in direction being faced & Critical  \\
\hline
\end{tabular}
\end{table}


\section*{Interactions}
\begin{table}[h]
\begin{tabular}{|l|L{4cm}|L{2cm}|L{3cm}|l|}
\hline
\textbf{Interaction} & \textbf{Trigger} & \textbf{Control}  & \textbf{Outcome} & \textbf{Importance} \\
\hline
Unsynchronized action & Attempt an action off-rhythm & Any action & Player dies & Critical \\
\hline 
Attack enemy & Occupy same space as an enemy & Move or jump & Enemy is defeated & Critical \\
\hline
Shoot enemy & Shot collides with enemy & Shoot & Enemy is defeated & Critical \\
\hline
Incorrect action & Player does not perform action specified by ticker & Any action other than the one specified & Player dies  & Critical \\
\hline
Invalid jump     & Attempt to jump with only one space available      & Jump   & Player moves forward instead & Valuable \\
\hline
Miss a beat      & Perform no action on a beat      & None       & Player dies       & Critical\\
\hline
\end{tabular}
\end{table}

\pagebreak
\section*{Challenges}
\subsection*{Matching specified actions}
When trying to match the actions that are given on the rhythm ticker,
players will have to plan their actions ahead of time. Plotting out a
path can be hard because if a player is off by only one beat, they
will lose the level. Figuring out just the right way to go so that the
appropriate actions are taken will take lots of practice from the
player.
\subsection*{Continuing movement once a level is started}
Once a level is started, the player must continue making one action
per beat until they reach a checkpoint or the end of the level. Since
the beat never stops, players must act quickly and will not have time
to think over moves so they need to plan ahead and be ready to change
their plan if something does not go as expected.
\subsection*{Avoiding Enemy Attacks}
Some enemies will have the ability to attack the player, so this
another thing they must keep in mind while planning out their route. A
path might seem good until the player gets hit with a projectile that
was shot by an enemy several squares away!

\end{document}

