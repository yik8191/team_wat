\documentclass[]{article}
\usepackage{amssymb,latexsym,amsmath}     % Standard packages
\usepackage[pdftex]{graphicx}
\usepackage{indentfirst}
\usepackage{pbox}
\usepackage{array}
\renewcommand{\baselinestretch}{1.05} 
\setlength\parindent{0pt}
\newcolumntype{L}[1]{>{
    \raggedright\let\newline\\\arraybackslash\hspace{0pt}}m{#1}
}


\addtolength{\textwidth}{1.0in}
\addtolength{\textheight}{1.00in}
\addtolength{\evensidemargin}{-0.75in}
\addtolength{\oddsidemargin}{-0.75in}
\addtolength{\topmargin}{-.50in}
\usepackage[lf]{venturis}
% \usepackage[T1]{fontenc}
% \usepackage{newtxtext,newtxmath}
% \renewcommand{\familydefault}{\sfdefault}  % sans-serif!

\usepackage{fancyhdr}
\renewcommand\headrule{}
\pagestyle{fancy}
\makeatletter
\let\ps@plain\ps@fancy % make plain pages fancy
\makeatother
\fancyhf{}
\lhead{\textbf{Rhythm Knights}\\
  Gameplay Prototype Report}
\rhead{\resizebox{0.75in}{!}{\includegraphics{img/logo_team.jpg}}}
\rfoot{\thepage}


\makeatletter
\renewcommand{\maketitle}{\bgroup\setlength{\parindent}{0pt}
\begin{flushleft}
  \huge{\textbf{\@title}}

  \large{\@author}
\end{flushleft}\egroup
}

\begin{document}

\title{\textbf{Rhythm Knights} \\ Technical Prototype Report}
\author{Team \emph{wat} \\
  \small{Kylar Henderson, Charles Tark, 
    Gagik Hakobyan, Julia Cole, Andrew Halpern, Austin Liu}}
\date{} % removed date
%\maketitle

\section*{Progress Report}
In our technical prototype, we implemented the core rhythm component
of the game by means of a central rhythm controller module.  In doing
this, we integrated the action ticker as well as a more fleshed out
gameplay structure.  We also established and began implementing the
modular architecture specification on which we will continue to build
our game.  In terms of art assets, more sprites were created for the
knight, enemies, ticker, background, and board tiles.  Additionally,
an initial music track was created.

\section*{Activity Breakdown}

\textbf{Charles Tark} \\
He was responsible for the implementation of the InputController
module, the health-related components, as well as various drawing code
throughout the different modules as necessary.  He was also
responsible for the creation of initial visual and audio assets for
the game. Specifically, he was involved with the following
activities:
\begin{itemize}
\item Implementing the InputController module (2 hours)
\item Implementing player HP in the Knight module (1 hour)
\item Implementing drawing of the health, ticker, and background (3 hours)
\item Creating visual assets for tiles and the ticker (3 hours)
\item Editing the initial music track produced by Austin (3 hours)
\item Meeting with the group (working on documents, 
  discussing game features/development) (8 hours)
\end{itemize}
In general, the activities listed were a necessary and productive use
of time.  He spent time performing the originally planned
tasks with the addition of helping to create the music track.  In
general, helping to create an adequately functioning rhythm element
proved to be more difficult than expected.  In the future, less time
would probably be spent on music.  However, some of the listed
activities were somewhat nonessential specifically for the production
of the technical prototype and could have been less prioritized, such
as the creation of new tile assets.\\

\textbf{Gagik Hakobyan} \\
He was responsible for creating the framework for our technical
prototype. This consisted of copying over code from the gameplay
prototype, as well as creating new classes according to our
architecture. He implemented the rhythm controller, as well as the
code in the gameplay controller that interacted with rhythm
controller. Finally, he was in charge of designing the game’s
architecture and creating the workflow and dependency diagrams. The
breakdown of hours was as follows:
\begin{itemize}
\item Designing the architecture (2 hours)
\item Creating workflow and dependency diagrams (1 hour)
\item Meeting with group members(working on documents, discussing game 
  features/development) (10 hours)
\item Implementing RhythmController (8 hours)
\item Writing parts of GameplayController that interface with 
  RhythmController (5 hours)
\end{itemize}
Overall, he spent around 26 hours on the game, which were mostly
productive. Rhythm controller took much longer to implement than
anticipated, and this took up a majority of his time.\\

\textbf{Austin Liu} \\
He was responsible for implementing the skeleton of the various
controller modules and composing preliminary music themes for the
game. Specifically, he worked on:
\begin{itemize}
\item Implementing skeleton code for the InputController interface and the 
  PlayerController and GameplayController classes that implement 
  InputController. (4 hours)
\item Implementing the bulk of the CollisionController module, which handles 
  bounds checking and collision detection between objects on the game board. 
  (4 hours)
\item Adding some skeleton code and abstract methods to the GameObject 
  abstract class. (2 hours)
\item Composing music for the game (3 hours)
\item Meeting to resolve programming and design decisions (4 hours)
\item Meeting to produce documents (4 hours)
\end{itemize}
In total, he spent about 21 hours on the game over the last two weeks. 
This was largely in line 
with the plan set out last week to spend 10 hours programming and 
4 hours on music.
The activities were largely a productive use of time, though he spent about 
1 to 2 hours resolving merge issues related to Git. These are becoming less 
of a problem.\\

\textbf{Kylar Henderson} \\
She was mostly responsible for all of the small coding miscellaneous tasks. 
She also spent time leading group meetings. Specifically her time was spent:
\begin{itemize}
\item Implementing all model classes (6-7 hours)
\item Implementing draw code for most of the game (1 hour)
\item Meeting with team members (10 hours)
\item Making minor tweaks to other group members' code (2 hours)
\end{itemize}
In total, she spent about 20 hours working on these. Almost all of
this time was productive and necessary to make the game functional for
the technical prototype. Almost all of the time that needed to be
spent was underestimated. All of the tasks ended up being more time
consuming than originally thought and certain tasks were not
considered, such as working on other group members' code. For the next
two weeks, time allotments will be more lenient, especially for major
tasks. \\

\textbf{Andrew Halpern} \\
He was responsible for designing the technical prototype game level
and creating the tile assets. He also worked with the rest of the team
to discuss gameplay mechanics and the way they will give feedback to
the player. Specifically, he was involved in:
\begin{itemize}
\item Designing the color scheme and tile assets (2 hours)
\item Designing the layouts for the initial gameplay levels and the 
  level demonstrated at technical prototype (3 hours)
\item Brainstorming new ideas for the gameplay mechanics with the 
  rest of the team (10 hours)
\item Creating a new background and start menu screen for the 
  technical prototype (2 hours)
\end{itemize}
In total, he spent around 16 hours working on these assets and
collaborating with the rest of the team for the technical
prototype. Designing levels for the technical prototype took a lot
longer than expected based on the complexity of the maps and
discussing with the rest of the team what mechanics would be feasible
for the technical prototype. Also, balancing the gameplay levels was
very challenging and proved to be too difficult for players in the
end. For future milestones, designing levels will be easier to test
and iterate on with a level editor. \\

\textbf{Julia Cole} \\
She was responsible for creating and updating the assets for the 
game and directing other members contributing assets. She was involved in:
\begin{itemize}
\item Creating an additional monster sprite (2 hours)
\item Creating a projectile sprite (10 minutes)
\item Altering the player sprite to be more cohesive with the 
  background (1 hour)
\item Creating ``dash'' player sprite (1.5 hours)
\item Directing Andrew and Charles on design tasks (1 hour)
\item Meeting with teammates (10 hours)
\end{itemize}
In total, she spent around fifteen hours working on the above
tasks. Workload will increase when a demand for animation
increases. Additional edits to the player sprite is needed, and the
werewolf sprite needs to be cleaned up as well as the skeleton
sprite. The dash sprite may need to be edited. Animations may include
basic walk cycles for each character asset. \\

\section*{Milestone Predictions}
For the alpha release, we plan to implement a level editor with a 
basic graphical user interface, as well as playable game level.
\subsection*{Tests for Acceptance}
Both designers, Andrew and Julia, are able to use the level editor to
design the board of the level. We should also be able to parse a JSON
representation of a level reliably so that we can test our level
editor GUI.\\

Additionally, we should implement one interesting level that showcases
all of the interesting features of our game and is considered as ``fun''
by at least 5 playtesters before our presentation.

\subsection*{Risk Assessment}
The following are potential risks we may face in regards to 
implementing our deliverables:
The JSON representation of a level is not easy to implement and the 
level editor module will likely require extensive debugging.
Synchronizing the RhythmController module with the music more carefully 
will likely require extensive testing to determine an optimal margin 
of error for the player with regard to the player's keyboard timings.

\subsection*{Activity Breakdown}

\noindent
\textbf{Charles Tark}\\
Over the next two weeks, Charles will be responsible for the
implementation of the revised health system and the incorporation of
visual and audio feedback for the players to reflect current
performance. Additionally, he will be responsible for the creation of
required visual assets as well as editing music tracks.
His rough time breakdown will be as follows:
\begin{itemize}
  \item Implementing the revised health system (2 to 3 hours)
  \item Implementing the visual and audio feedback in code (4 to 5 hours)
  \item Producing necessary visual and audio feedback assets (3 to 4 hours)
  \item Editing music tracks (2 to 3 hours)
  \item Producing of any other needed visual assets (2 to 4 hours)
  \item Meeting with team members (4 - 6 hours)
\end{itemize}

\noindent
\textbf{Gagik Hakobyan}
Over the next two weeks, Gagik will be responsible for fleshing out
the rhythm controller to allow us more precise control over action
timings. This will allow us to implement more complex actions, such a
dashing and casting freeze, which he will be in charge of implementing
as well. The anticipated activity breakdown is as follows:
\begin{itemize}
\item Revising rhythm controller (6 - 7 hours)
\item Implementing dash, fireball, and freeze (4 - 5 hours)
\item Assigning tasks and working with programmers to integrate their 
  code (1 - 2 hours)
\item Meeting with team members (4 - 6 hours)
\end{itemize}

\noindent
\textbf{Austin Liu}\\
Over the next two weeks, Austin will be responsible for helping Gagik
with improving the quality of the audio visual feedback of the
game. His rough time breakdown will be as follows:
\begin{itemize}
\item Composing and arranging more music, as well as looking into the 
  JFugue third party library for writing music in Java to avoid music format 
  issues (6 hours)
\item Working in the code required in the CollisionController and 
  GameplayController to handle additional additional game actions such 
  as dashing and projectiles (3 hours)
\item Working with Gagik to improve the robustness of the RhythmController 
  by providing information on how to anticipate beats in music using various 
  methods (6 hours)
\item Meeting with the group (4-5 hours)
\end{itemize}

\noindent
\textbf{Kylar Henderson}\\
Over the next two weeks, Kylar will be responsible for leading the
work on the level editor. Her rough time breakdown will be as follows:
\begin{itemize}
\item Researching JSON for use by the level editor and loading in levels 
  (4-5 hours)
\item Implementing a parser that will convert JSON files into levels (6-7 hours)
\item Implementing a GUI for players to use to design levels (7-8 hours)
\end{itemize}

\noindent
\textbf{Andrew Halpern}
Over the next two weeks, Andrew will be responsible for redesigning
visual assets to help players follow along with the game's
rhythm. This includes environmental assets such as tiles and
background. Andrew will also be responsible for designing the way the
new combo bar and rhythm movement status bar will interact and
feel. He will also begin to brainstorm and create UI elements that
help teach the players how the core gameplay mechanics work in the
beginning levels. His rough time breakdown will be as follows:
\begin{itemize}
\item Designing more gameplay levels feature new gameplay mechanics such as 
  dashing and testing the level editor (5 hours)
\item Collaborating with other team members to create the level editor (2 hours)
\item Redesigning tile assets to give player more feedback (3 hours)
\item Creating new ``ticker'' and ``combo'' meter to reflect changes from 
  technical prototype   (5 hours)
\item Brainstorming and meeting with rest of the team (10 hours)
\end{itemize}

\noindent
\textbf{Julia Cole}
Over the next two weeks, Julia will be responsible for directing the
production of assets. She will continue to refine character design,
implement animation for all characters and tweak existing assets.
Her rough time breakdown will be as follows:
\begin{itemize}
\item Polishing character drawings (3 hours)
\item Animating walk cycles (10 hours)
\item Tweaking graphic design assets (1 hour)
\item Checking in with Charles and Andrew to ensure a cohesive design (1 hour)
\item Meeting with the team (4-6 hours)
\end{itemize}
\end{document}

