\documentclass[]{article}
\usepackage{amssymb,latexsym,amsmath}     % Standard packages
\usepackage[pdftex]{graphicx}
\usepackage{indentfirst}
\usepackage{pbox}
\usepackage{array}
\renewcommand{\baselinestretch}{1.05} 
\setlength\parindent{0pt}
\newcolumntype{L}[1]{>{
    \raggedright\let\newline\\\arraybackslash\hspace{0pt}}m{#1}
}


\addtolength{\textwidth}{1.0in}
\addtolength{\textheight}{1.00in}
\addtolength{\evensidemargin}{-0.75in}
\addtolength{\oddsidemargin}{-0.75in}
\addtolength{\topmargin}{-.50in}
\usepackage[lf]{venturis}
% \usepackage[T1]{fontenc}
% \usepackage{newtxtext,newtxmath}
% \renewcommand{\familydefault}{\sfdefault}  % sans-serif!

\usepackage{fancyhdr}
\renewcommand\headrule{}
\pagestyle{fancy}
\makeatletter
\let\ps@plain\ps@fancy % make plain pages fancy
\makeatother
\fancyhf{}
\lhead{\textbf{Rhythm Knights}\\
  Gameplay Prototype Report}
\rhead{\resizebox{0.75in}{!}{\includegraphics{img/logo_team.jpg}}}
\rfoot{\thepage}


\makeatletter
\renewcommand{\maketitle}{\bgroup\setlength{\parindent}{0pt}
\begin{flushleft}
  \huge{\textbf{\@title}}

  \large{\@author}
\end{flushleft}\egroup
}

\begin{document}

\title{\textbf{Rhythm Knights} \\ Gameplay Prototype Report}
\author{Team \emph{wat} \\
  \small{Kylar Henderson, Charles Tark, 
    Gagik Hakobyan, Julia Cole, Andrew Halpern, Austin Liu}}
\date{} % removed date
%\maketitle

\section*{Progress Report}
In our gameplay prototype, we demonstrated basic two-dimensional,
grid-based movement on a set of tiles corresponding to the first
level.  Furthermore, we implemented enemy movement that is performed
in step to the players' actions, that is, the enemies move only when
the players do.  In terms of assets, basic sprites were created for
the knight, enemies, ticker, and board tiles.

\section*{Activity Breakdown}

\textbf{Charles Tark} 
was responsible for implementation of the InputController module as
well as ensuring that controls and player movement functioned as
expected.  He was also responsible for the synchronizing the movements
of enemies with the players'.  Specifically, he was involved with the
following activities:
\begin{itemize}
\item Implementation of the InputController module
\item Debugging of knight movement
\item Debugging of the enemy movement
\item Synchronizing enemy movement with player movement
\item Creation of the game logo
\end{itemize}

In total, he spent approximately 9 hours working on the associated
tasks.  In general, the activities listed were a necessary and
productive use of time.  However, a lot of time (around 1-2 hours) was
spent configuring the IDE and Git.  In the future, minimal time will
be spent performing such tasks.\\

\textbf{Gagik Hakobyan} 
was responsible for importing the framework for our gameplay
prototype. He combined elements of labs 2 and 3 into a single skeleton
with TODOs for each group member to implement. He also spent some time
with Austin and Charles discussing the architecture of both our game
and the gameplay prototype. Specifically he:
\begin{itemize}
\item Built the project skeleton/framework
\item Determined a rough software architecture
\item Wrote initialization code for the level
\item Assigned rough programming tasks to everyone in the group
\end{itemize}
In total, he spent around 9 hours working on writing the skeleton and
other code associated with the gameplay prototype and an additional
2-3 hours discussing the architecture with Austin and Charles.\\

\textbf{Austin Liu} 
was responsible for implementing some modules and working on 
certain controller modules with Gagik. Specifically, he was involved in:
\begin{itemize}
\item Implementing the Knight module
\item Implementing bounds in the character movement
\item Composing and arranging sample music files 
\item Setting up and maintaining the Git repository, as well as well 
  as helping resolve configuration issues between Git and IDEs. 
\end{itemize}
In total, he spent about 9 hours working on the above tasks. However,
at least 2-3 hours were spent resolving issues related to branching
and merging on Git using both the command line and graphical
interfaces. These issues should become less problematic now that the
.gitignore file has been set up.\\

\textbf{Kylar Henderson} 
was responsible for implementing drawing code for all 
elements of the prototype and keeping track of team progress. 
Her main tasks were:
\begin{itemize}
\item Establishing draw methods for the Knight, Enemies, and Board
\item Converting enemy and knight positions to on screen coordinates
\item Collaborating with other team members to distribute work 
\end{itemize}
In total, she spent about 5 hours working on these. Most of this time
was spent working with other team members to coordinate plans. In the
future though, more time will be committed to working on programming.\\

\textbf{Andrew Halpern} 
was responsible for designing the game level for and 
background for the first level.
Specifically, he was involved in:
\begin{itemize}
\item Designing the tiles for the first level
\item Designing the game's rhythm ``ticker'' to show player's status in the game
\item Brainstorming new ideas for the game's core mechanics
\end{itemize}
In total, he spent around six hours working on these assets and 
collaborating with the rest of the team for the gameplay prototype. \\

\textbf{Julia Cole} 
was responsible for creating assets for the game and 
directing other members contributing assets. She was involved in:
\begin{itemize}
\item Creating the player sprite
\item Creating the monster sprite
\item Directing Andrew and Charles on design tasks
\end{itemize}
In total, she spent around four hours working on the above tasks. 
It is expected that the workload for design will increase when 
we settle on the details of our game. \\

Additionally, the group spent about two hours last week and 
four hours this week on group discussion.

\section*{Milestone Predictions}
For the technical prototype, we plan to incorporate the rhythm
component into our current prototype, such that movement on a rhythm
is viable in the style of the puzzler we are considering.
Furthermore, we plan to implement and integrate the action ticker into
our prototype as well as win and loss states.  During this time, we
will consider and experiment with various rhythm-based gameplay
schemes such that gameplay is not too hectic and/or challenging.  We
will also complete the architectural specification.
\subsection*{Tests for Acceptance}
\begin{itemize}
\item The rhythm controller doesn't desynchronize noticeably
\item Win and loss states function properly and as expected
\item The action ticker correctly indicates the next required actions
\item Rhythm-based gameplay is integrated such that it is not too 
  difficult or frustrating
\item Players receive visual and/or auditory feedback for 
  successfully performing certain actions such as matching the rhythm
\item The board will be implemented with tiles in a way that is simple
\end{itemize}
\subsection*{Risk Assessment}
\begin{itemize}
\item Music may be difficult to synchronize properly with the 
  actions of all characters
\item Identifying an adequate scheme by which we would 
  integrate rhythm may prove to be challenging
\item Certain players may find the combination of rhythm and 
  puzzle elements to be too difficult or complicated
\end{itemize}
\subsection*{Activity Breakdown}

We intend to allocate 4 hours over the next two weeks to meeting time.\\

\noindent
\textbf{Charles Tark}
Over the next two weeks, Charles will be responsible for helping Gagik
in the creation of a new architecture specification, implementing and
debugging any modules necessary for the technical prototype, and
creating any necessary visual assets as specified.
The time estimates are as follows:
\begin{itemize}
  \item Designing the game's architecture with Gagik - 4 to 5 hours
  \item Miscellaneous coding necessary - 10 to 12 hours
  \item Production of any needed visual assets - 2 to 4 hours
\end{itemize}

\noindent
\textbf{Gagik Hakobyan}
Over the next two weeks, Gagik will be responsible for designing the
game’s architecture with Charles, and more specifically with designing
the skeleton for our technical prototype. He will also help in
implementing our rhythm system.
The time estimates are as follows:
\begin{itemize}
  \item Designing the game's architecture with Gagik - 4 to 5 hours
  \item Miscellaneous coding necessary - 10 to 12 hours
  \item Production of any needed visual assets - 2 to 4 hours
\end{itemize}

\noindent
\textbf{Austin Liu}
Over the next two weeks, Austin will be responsible for helping Gagik with the 
The time estimates are as follows:
\begin{itemize}
\item Music composition and preparation - 4 hours
\item Helping Gagik with modules related to rhythm detection 
  and synchronization - 3 to 4 hours 
\item Miscellaneous programming assigned by Gagik - 10 to 12 hours

\noindent
\end{itemize}
\textbf{Kylar Henderson}
Over the next two weeks, Kylar will be responsible for working on coding 
tasks assigned by Gagik and managing team meetings and progress. 
The time estimates are as follows:
\begin{itemize}
\item Miscellaneous programming tasks assigned by Gagik - 8 to 10 hours
\item Coordinating and leading group meetings - 4 to 5 hours
\end{itemize}

\noindent
\textbf{Andrew Halpern}
Over the next two weeks, Andrew will be responsible for designing more
gameplay levels, creating more tile assets for the game, and
redesigning the new ``ticker'' meter graphics
The time estimates are as follows:
\begin{itemize}
\item Gameplay Levels - 3 hours
\item Tile Assets - 3 hours
\item New ``ticker'' meter design - 3 hours
\end{itemize}

\noindent
\textbf{Julia Cole}
Over the next two weeks, Julia will be responsible for directing the 
production of assets. She will refine character design, attempt some 
basic animation, and ensure that everything is connecting visually.
The time estimates are as follows:
\begin{itemize}
\item Polish and add any necessary characters - 2 to 4 hours
\item Implement basic animation - 2 to 4 hours
\item Check in with Charles and Andrew and ensure a cohesive design - 1 hour
\end{itemize}
\end{document}

